\chapter{Arhitektura i dizajn sustava}
		
		\textbf{\textit{dio 1. revizije}}\\

		\textit{ Potrebno je opisati stil arhitekture te identificirati: podsustave, preslikavanje na radnu platformu, spremišta podataka, mrežne protokole, globalni upravljački tok i sklopovsko-programske zahtjeve. Po točkama razraditi i popratiti odgovarajućim skicama:}
	\begin{itemize}
		\item 	\textit{izbor arhitekture temeljem principa oblikovanja pokazanih na predavanjima (objasniti zašto ste baš odabrali takvu arhitekturu)}
		\item 	\textit{organizaciju sustava s najviše razine apstrakcije (npr. klijent-poslužitelj, baza podataka, datotečni sustav, grafičko sučelje)}
		\item 	\textit{organizaciju aplikacije (npr. slojevi frontend i backend, MVC arhitektura) }		
	\end{itemize}
	
				
		\section{Baza podataka}
			
			\textbf{\textit{dio 1. revizije}}\\
			
		\textit{Potrebno je opisati koju vrstu i implementaciju baze podataka ste odabrali, glavne komponente od kojih se sastoji i slično.}
		
		Za potrebe razvoja \textit{Gigera} koristit će se objektno relacijsko mapiranje. To je metoda koja se koristi u objektno-orijentiranim jezicima te se na taj način stvara virtualna objektna baza podataka.
		
		Baza podataka sastoji se od sljedećih entiteta:
		
		\begin{packed_item}
			\item Review
			\item Review\_gig
			\item Review\_band
			\item Review\_musician
			\item Review\_organizer
			\item Organizer
			\item Users
			\item Conversation\_user
			\item Conversation
			\item Message
			\item Message\_seen
			\item Musician
			\item Instruments
			\item Musician\_bands
			\item Band
			\item Band\_members
			\item Band\_gigs
			\item Musician\_gig
			\item Gig
			\item Gig\_type
			\item Post
			\item Comment
		\end{packed_item}
			
		
		
			\subsection{Opis tablica}
			

				\textit{Svaku tablicu je potrebno opisati po zadanom predlošku. Lijevo se nalazi točno ime varijable u bazi podataka, u sredini se nalazi tip podataka, a desno se nalazi opis varijable. Svjetlozelenom bojom označite primarni ključ. Svjetlo plavom označite strani ključ}
				
				\begin{longtabu} to \textwidth {|X[6, l+3]|X[6, l]|X[20, l]|}
					
					\hline \multicolumn{3}{|c|}{\textbf{Review}}	 \\[3pt] \hline
					\endfirsthead
					
					\hline 
					\endlastfoot
					
					id & BIGINT	&  	jedinstveni identifikator recenzije 	\\ \hline
					content	& VARCHAR &  sadržaj komentara	\\ \hline 
					grade & INT & ocjena 1-5  \\ \hline 
					author\_id & BIGINT	& jedinstveni identifikator autora recenzije	\\ \hline 
					
					
				\end{longtabu}

			  \begin{longtabu} to \textwidth {|X[6, l+3]|X[6, l]|X[20, l]|}
				
				\hline \multicolumn{3}{|c|}{\textbf{Review\_gig}}	 \\[3pt] \hline
				\endfirsthead
				
				\hline 
				\endlastfoot
				
				fk\_gig & BIGINT	&  	jedinstveni identifikator nastupa 	\\ \hline
				fk\_review	& BIGINT &  jedinstveni identifikator recenzije	\\ \hline 		
				
			\end{longtabu}	
		
		\begin{longtabu} to \textwidth {|X[6, l+3]|X[6, l]|X[20, l]|}
			
			\hline \multicolumn{3}{|c|}{\textbf{Review\_band}}	 \\[3pt] \hline
			\endfirsthead
			
			\hline 
			\endlastfoot
			
			fk\_band & BIGINT	&  	jedinstveni identifikator benda 	\\ \hline
			fk\_review	& BIGINT &  jedinstveni identifikator recenzije \\ \hline 		
			
		\end{longtabu}		
	
	\begin{longtabu} to \textwidth {|X[6, l+3]|X[6, l]|X[20, l]|}
		
		\hline \multicolumn{3}{|c|}{\textbf{Review\_musician}}	 \\[3pt] \hline
		\endfirsthead
		
		\hline 
		\endlastfoot
		
		fk\_musician & BIGINT	&  	jedinstveni identifikator glazbenika 	\\ \hline
		fk\_review	& BIGINT &  jedinstveni identifikator recenzije	\\ \hline 		
		
	\end{longtabu}	

	\begin{longtabu} to \textwidth {|X[6, l+3]|X[6, l]|X[20, l]|}
		
		\hline \multicolumn{3}{|c|}{\textbf{Review\_organizer}}	 \\[3pt] \hline
		\endfirsthead
		
		\hline 
		\endlastfoot
		
		fk\_organizer & BIGINT	&  	jedinstveni identifikator organizatora 	\\ \hline
		fk\_review	& BIGINT &  jedinstveni identifikator recenzije	\\ \hline 		
		
	\end{longtabu}	

	\begin{longtabu} to \textwidth {|X[6, l+3]|X[6, l]|X[20, l]|}
		
		\hline \multicolumn{3}{|c|}{\textbf{Organizer}}	 \\[3pt] \hline
		\endfirsthead
		
		\hline 
		\endlastfoot
		
		id & BIGINT	&  	jedinstveni identifikator organizatora 	\\ \hline
		manager\_name	& VARCHAR &  ime organizatora	\\ \hline 		
		
	\end{longtabu}	

	\begin{longtabu} to \textwidth {|X[6, l+3]|X[6, l]|X[20, l]|}
		
		\hline \multicolumn{3}{|c|}{\textbf{User}}	 \\[3pt] \hline
		\endfirsthead
		
		\hline 
		\endlastfoot
		
		id & BIGINT	&  	jedinstveni identifikator korisnika 	\\ \hline
		email & VARCHAR & email adresa korisnika	\\ \hline 		
		password\_hash & VARCHAR & 
		hash lozinke korisnika \\ \hline
		phone\_number & VARCHAR & telefonski broj korisnika \\ \hline
		musician\_id & BIGINT & jedinstveni identifikator glazbenika 
	\end{longtabu}	

	\begin{longtabu} to \textwidth {|X[6, l+3]|X[6, l]|X[20, l]|}
		
		\hline \multicolumn{3}{|c|}{\textbf{Conversation\_user}}	 \\[3pt] \hline
		\endfirsthead
		
		\hline 
		\endlastfoot
		
		fk\_user & BIGINT	&  	jedinstveni identifikator korisnika	\\ \hline
		fk\_conversation	& BIGINT &  jedinstveni identifikator razgovora	\\ \hline 		
		
	\end{longtabu}

	\begin{longtabu} to \textwidth {|X[6, l+3]|X[6, l]|X[20, l]|}
		
		\hline \multicolumn{3}{|c|}{\textbf{Conversation}}	 \\[3pt] \hline
		\endfirsthead
		
		\hline 
		\endlastfoot
		
		id & BIGINT	&  	jedinstveni identifikator razgovora 	\\ \hline
		name	& VARCHAR &  ime razgovora	\\ \hline 		
		
	\end{longtabu}

	\begin{longtabu} to \textwidth {|X[6, l+3]|X[6, l]|X[20, l]|}
		
		\hline \multicolumn{3}{|c|}{\textbf{Message}}	 \\[3pt] \hline
		\endfirsthead
		
		\hline 
		\endlastfoot
		
		id & BIGINT	&  	jedinstveni identifikator poruke 	\\ \hline
		content	& VARCHAR & sadržaj poruke	\\ \hline 	
		sent\_time & TIMESTAMP & vrijeme kada je poruka poslana \\ \hline
		fk\_sender & BIGINT & jedinstveni identifikator pošiljatelja \\ \hline
		fk\_converation & BIGINT & jedinstveni identifikator razgovora \\ \hline	
		
	\end{longtabu}

	\begin{longtabu} to \textwidth {|X[6, l+3]|X[6, l]|X[20, l]|}
		
		\hline \multicolumn{3}{|c|}{\textbf{Message\_seen}}	 \\[3pt] \hline
		\endfirsthead
		
		\hline 
		\endlastfoot
		
		fk\_message & BIGINT	&  	jedinstveni identifikator poruke 	\\ \hline
		fk\_user	& BIGINT &  jedinstveni identifikator korisnika	\\ \hline 		
		
	\end{longtabu}

	\begin{longtabu} to \textwidth {|X[6, l+3]|X[6, l]|X[20, l]|}
		
		\hline \multicolumn{3}{|c|}{\textbf{Musician}}	 \\[3pt] \hline
		\endfirsthead
		
		\hline 
		\endlastfoot
		
		id & BIGINT	&  	jedinstveni identifikator glazbenika 	\\ \hline
		bio	& VARCHAR &  opis glazbenika	\\ \hline 		
		
	\end{longtabu}

	\begin{longtabu} to \textwidth {|X[6, l+3]|X[6, l]|X[20, l]|}
		
		\hline \multicolumn{3}{|c|}{\textbf{Instruments}}	 \\[3pt] \hline
		\endfirsthead
		
		\hline 
		\endlastfoot
		
		musician & BIGINT	&  	jedinstveni identifikator glazbenika	\\ \hline
		plays	& VARCHAR &  ime instrumenta	\\ \hline 		
		
	\end{longtabu}

	\begin{longtabu} to \textwidth {|X[6, l+3]|X[6, l]|X[20, l]|}
		
		\hline \multicolumn{3}{|c|}{\textbf{Musician\_bands}}	 \\[3pt] \hline
		\endfirsthead
		
		\hline 
		\endlastfoot
		
		fk\_musician & BIGINT	&  	jedinstveni identifikator glazbenika 	\\ \hline
		fk\_band	& BIGINT &  jedinstveni identifikator benda	\\ \hline 		
		
	\end{longtabu}

	\begin{longtabu} to \textwidth {|X[6, l+3]|X[6, l]|X[20, l]|}
		
		\hline \multicolumn{3}{|c|}{\textbf{Band}}	 \\[3pt] \hline
		\endfirsthead
		
		\hline 
		\endlastfoot
		
		id & BIGINT	&  	jedinstveni identifikator benda 	\\ \hline
		bio & VARCHAR & opis benda \\ \hline
		formed\_date & DATE & datum osnutka benda \\ \hline
		name & VARCHAR & ime benda \\ \hline
		leader\_id	& BIGINT &  jedinstveni identifikator voditelja benda	\\ \hline 	
		
	\end{longtabu}
		
		\begin{longtabu} to \textwidth {|X[6, l+3]|X[6, l]|X[20, l]|}
			
			\hline \multicolumn{3}{|c|}{\textbf{Band\_members}}	 \\[3pt] \hline
			\endfirsthead
			
			\hline 
			\endlastfoot
			
			band\_id & BIGINT	&  	jedinstveni identifikator benda 	\\ \hline
			members\_id	& BIGINT &  jedinstveni identifikator člana benda	\\ \hline 		
			
		\end{longtabu}	
	
	\begin{longtabu} to \textwidth {|X[6, l+3]|X[6, l]|X[20, l]|}
		
		\hline \multicolumn{3}{|c|}{\textbf{Band\_gigs}}	 \\[3pt] \hline
		\endfirsthead
		
		\hline 
		\endlastfoot
		
		band\_id & BIGINT	&  	jedinstveni identifikator benda 	\\ \hline
		gigs\_id	& BIGINT &  jedinstveni identifikator nastupa	\\ \hline 		
		
	\end{longtabu}	

	\begin{longtabu} to \textwidth {|X[6, l+3]|X[6, l]|X[20, l]|}
		
		\hline \multicolumn{3}{|c|}{\textbf{Musician\_gig}}	 \\[3pt] \hline
		\endfirsthead
		
		\hline 
		\endlastfoot
		
		fk\_musician & BIGINT	&  	jedinstveni identifikator glazbenika 	\\ \hline
		fk\_gig	& BIGINT &  jedinstveni identifikator nastupa	\\ \hline 		
		
	\end{longtabu}	

	\begin{longtabu} to \textwidth {|X[6, l+3]|X[6, l]|X[20, l]|}
		
		\hline \multicolumn{3}{|c|}{\textbf{Gig}}	 \\[3pt] \hline
		\endfirsthead
		
		\hline 
		\endlastfoot
		
		id & BIGINT	&  	jedinstveni identifikator nastupa 	\\ \hline
		date\_time & TIMESTAMP & datum i vrijeme održavanja nastupa \\ \hline
		description & VARCHAR & opis nastupa \\ \hline
		expected\_duration & VARCHAR & očekivano trajanje nastupa \\ \hline
		final\_deal\_achieved & BOOLEAN & dogovor postignut ili ne \\ \hline
		gig\_type & INT & vrsta nastupa \\ \hline
		address & VARCHAR & adresa održavanja nastupa \\ \hline
		extra\_description & VARCHAR & dodatan opis nastupa \\ \hline
		x & DOUBLE & ?? \\ \hline
		y & DOUBLE & ?? \\ \hline
		private\_gig & BOOLEAN & nastupa privatan ili ne \\ \hline
		proposed\_price & INT & preporučena cijena ulaznice \\ \hline
		final\_band\_id	& BIGINT &  jedinstveni identifikator benda	\\ \hline
		organizer\_id	& BIGINT &  jedinstveni identifikator organizatora	\\ \hline 		
		
	\end{longtabu}	

		\begin{longtabu} to \textwidth {|X[6, l+3]|X[6, l]|X[20, l]|}
		
		\hline \multicolumn{3}{|c|}{\textbf{Gig\_type}}	 \\[3pt] \hline
		\endfirsthead
		
		\hline 
		\endlastfoot
		
		gig &  BIGINT	&  	jedinstveni identifikator vrste nastupa 	\\ \hline
		gig\_type	& VARCHAR &  vrsta nastupa	\\ \hline 		
		
	\end{longtabu}

		\begin{longtabu} to \textwidth {|X[6, l+3]|X[6, l]|X[20, l]|}
		
		\hline \multicolumn{3}{|c|}{\textbf{Post}}	 \\[3pt] \hline
		\endfirsthead
		
		\hline 
		\endlastfoot
		
		id & BIGINT	&  	jedinstveni identifikator objave 	\\ \hline
		content & VARCHAR & sadržaj objave \\ \hline
		published\_on & TIMESTAMP & datum i vrijeme objave objave \\ \hline	
		fk\_user & BIGINT & jedinstveni identifikator korisnika koji je napisao objavu \\ \hline
		fk\_band & BIGINT & jedinstveni identifikator benda \\ \hline
			
	\end{longtabu}

	\begin{longtabu} to \textwidth {|X[6, l+3]|X[6, l]|X[20, l]|}
		
		\hline \multicolumn{3}{|c|}{\textbf{Comment}}	 \\[3pt] \hline
		\endfirsthead
		
		\hline 
		\endlastfoot
		
		id & BIGINT	&  	jedinstveni identifikator komentara 	\\ \hline
		content & VARCHAR & sadržaj komentara \\ \hline
		posted\_on & TIMESTAMP & datum i vrijeme objave komentara \\ \hline	
		author\_id & BIGINT & jedinstveni identifikator autora komentara \\ \hline
		fk\_post & BIGINT & jedinstveni identifikator komentara \\ \hline
		
	\end{longtabu}
	

		
	

			
			\subsection{Dijagram baze podataka}
				\textit{ U ovom potpoglavlju potrebno je umetnuti dijagram baze podataka. Primarni i strani ključevi moraju biti označeni, a tablice povezane. Bazu podataka je potrebno normalizirati. Podsjetite se kolegija "Baze podataka".}
			
			\eject
			
			
		\section{Dijagram razreda}
		
			\textit{Potrebno je priložiti dijagram razreda s pripadajućim opisom. Zbog preglednosti je moguće dijagram razlomiti na više njih, ali moraju biti grupirani prema sličnim razinama apstrakcije i srodnim funkcionalnostima.}\\
			
			\textbf{\textit{dio 1. revizije}}\\
			
			\textit{Prilikom prve predaje projekta, potrebno je priložiti potpuno razrađen dijagram razreda vezan uz \textbf{generičku funkcionalnost} sustava. Ostale funkcionalnosti trebaju biti idejno razrađene u dijagramu sa sljedećim komponentama: nazivi razreda, nazivi metoda i vrste pristupa metodama (npr. javni, zaštićeni), nazivi atributa razreda, veze i odnosi između razreda.}\\
			
			\textbf{\textit{dio 2. revizije}}\\			
			
			\textit{Prilikom druge predaje projekta dijagram razreda i opisi moraju odgovarati stvarnom stanju implementacije}
			
			
			
			\eject
		
		\section{Dijagram stanja}
			
			
			\textbf{\textit{dio 2. revizije}}\\
			
			\textit{Potrebno je priložiti dijagram stanja i opisati ga. Dovoljan je jedan dijagram stanja koji prikazuje \textbf{značajan dio funkcionalnosti} sustava. Na primjer, stanja korisničkog sučelja i tijek korištenja neke ključne funkcionalnosti jesu značajan dio sustava, a registracija i prijava nisu. }
			
			
			\eject 
		
		\section{Dijagram aktivnosti}
			
			\textbf{\textit{dio 2. revizije}}\\
			
			 \textit{Potrebno je priložiti dijagram aktivnosti s pripadajućim opisom. Dijagram aktivnosti treba prikazivati značajan dio sustava.}
			
			\eject
		\section{Dijagram komponenti}
		
			\textbf{\textit{dio 2. revizije}}\\
		
			 \textit{Potrebno je priložiti dijagram komponenti s pripadajućim opisom. Dijagram komponenti treba prikazivati strukturu cijele aplikacije.}