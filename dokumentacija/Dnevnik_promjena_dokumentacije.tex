\chapter{Dnevnik promjena dokumentacije}
		
		\textbf{\textit{Kontinuirano osvježavanje}}\\
				
		
		\begin{longtabu} to \textwidth {|X[2, l]|X[13, l]|X[3, l]|X[3, l]|}
			\hline \multicolumn{1}{|l|}{\textbf{Rev.}}	& \multicolumn{1}{l|}{\textbf{Opis promjene/dodatka}} & \multicolumn{1}{|l|}{\textbf{Autori}} & \multicolumn{1}{l|}{\textbf{Datum}} \\[3pt] \hline
			\endfirsthead
			
			\hline \multicolumn{1}{|l|}{\textbf{Rev.}}	& \multicolumn{1}{l|}{\textbf{Opis promjene/dodatka}} & \multicolumn{1}{|l|}{\textbf{Autori}} & \multicolumn{1}{l|}{\textbf{Datum}} \\[3pt] \hline
			\endhead
			
			\hline 
			\endlastfoot
			
			0.1 & Napravljen predložak, modificirana \newline glavna .tex datoteka.	& Lanča & 26.10.2019. 		\\[3pt] \hline 
			0.2	& Unesen dnevnik sastanaka. & Lanča & 27.10.2019. 	\\[3pt] \hline 
			0.3 & Dodani \textit{Use Case} dijagrami br. 2, 3, 4, 5, 6 & Jurić & 29.10.2019. \\[3pt] \hline
			0.4 & Dodan \textit{Use Case} dijagram br. 9 & Nosil & 29.10.2019. \\[3pt] \hline
			0.5 & Dodani \textit{Use Case} dijagrami br. 10, 11, 12, 13, \newline 17, 18 & Gaši & 29.10.2019. \\[3pt] \hline
			0.6 & Dodani \textit{Use Case} dijagrami br. 1, 7, 8, 14, 15, \newline 16, 19, 20 & Zec & 30.10.2019. \\[3pt] \hline
			0.7 & Dodani funkcionalni zahtjevi & Zec &         04.11.2019. \\[3pt] \hline 
			0.8 & Nadopunjen zapisnik sastanka, dodani ostali zahtjevi & Lanča & 04.11.2019. \\[3pt] \hline 
			0.9 & Dodan opis projektnog zadatka & Gaši & 05.11.2019. \\[3pt] \hline 
			0.10 & Dodani \textit{Use Case} dijagrami br. 21-35& Zec & 07.11.2019. \\[3pt] \hline 
			0.11 & Rastavljeni neki \textit{Use Case}-ovi na više njih & Gaši & 08.11.2019. \\[3pt] \hline 
			0.12.1 & Dodane opisne tablice baze & Gaši & 08.11.2019 \\[3pt] \hline 
			0.12.2 & Uneseni sekvencijski dijagrami njihov opis & Jurić, Gaši & 12.11.2019. \\[3pt] \hline 
			0.13 & Unesen opis arhitekture & Lanča & 12.11.2019. \\[3pt] \hline 
			0.14 & Unesen opis tablica & Krmek & 12.11.2019. \\[3pt] \hline 
			0.15 & Dodani dijagrami obrazaca uporabe & Zec & 12.11.2019. \\[3pt] \hline 
			0.16 & Promjenjeni opisi aktora i use case-va 1 do 15 & Nosil & 12.11.2019. \\[3pt] \hline 
			0.17 & Promjenjeni Use case-ovi 16 do 36 & Nosil & 12.11.2019. \\[3pt] \hline 
			\textbf{1.0} & Verzija samo s bitnim dijelovima za 1. ciklus & Ivošević & 11.09.2013. \\[3pt] \hline 
			1.1 & Uređivanje teksta -- funkcionalni i nefunkcionalni zahtjevi & Grudenić \newline Jović & 14.09.2013. \\[3pt] \hline 
			1.2 & Manje izmjene:Timer - Brojilo vremena & Grudenić & 15.09.2013. \\[3pt] \hline 
			1.3 & Popravljeni dijagrami obrazaca uporabe & Jović & 15.09.2013. \\[3pt] \hline 
			1.5 & Generalna revizija strukture dokumenta & Ivošević & 19.09.2013. \\[3pt] \hline 
			1.5.1 & Manja revizija (dijagram razmještaja) & Jović & 20.09.2013. \\[3pt] \hline 
			\textbf{2.0} & Konačni tekst predloška dokumentacije  & Ivošević & 28.09.2013. \\[3pt] \hline 
			&  &  & \\[3pt] \hline
			
			
		\end{longtabu}
	
	
		\textit{Moraju postojati glavne revizije dokumenata 1.0 i 2.0 na kraju prvog i drugog ciklusa. Između tih revizija mogu postojati manje revizije već prema tome kako se dokument bude nadopunjavao. Očekuje se da nakon svake značajnije promjene (dodatka, izmjene, uklanjanja dijelova teksta i popratnih grafičkih sadržaja) dokumenta se to zabilježi kao revizija. Npr., revizije unutar prvog ciklusa će imati oznake 0.1, 0.2, …, 0.9, 0.10, 0.11.. sve do konačne revizije prvog ciklusa 1.0. U drugom ciklusu se nastavlja s revizijama 1.1, 1.2, itd.}