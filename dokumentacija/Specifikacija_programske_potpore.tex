\chapter{Specifikacija programske potpore}
		
	\section{Funkcionalni zahtjevi}
			
			\textbf{\textit{dio 1. revizije}}\\
			
			\textit{Navesti \textbf{dionike} koji imaju \textbf{interes u ovom sustavu} ili  \textbf{su nositelji odgovornosti}. To su prije svega korisnici, ali i administratori sustava, naručitelji, razvojni tim.}\\
				
			\textit{Navesti \textbf{aktore} koji izravno \textbf{koriste} ili \textbf{komuniciraju sa sustavom}. Oni mogu imati inicijatorsku ulogu, tj. započinju određene procese u sustavu ili samo sudioničku ulogu, tj. obavljaju određeni posao. Za svakog aktora navesti funkcionalne zahtjeve koji se na njega odnose.}\\
			
			
			\noindent \textbf{Dionici:}
			
			\begin{packed_enum}
				
				\item Dionik 1
				\item Dionik 2				
				\item ...
				
			\end{packed_enum}
			
			\noindent \textbf{Aktori i njihovi funkcionalni zahtjevi:}
			
			
			\begin{packed_enum}
				\item  \underbar{Aktor 1 (inicijator) može:}
				
				\begin{packed_enum}
					
					\item funkcionalnost 1
					\item funkcionalnost 2
					\begin{packed_enum}
						
						\item  podfunkcionalnost 1 
						\item  podfunkcionalnost 2
				
					\end{packed_enum}
					\item  funkcionalnost 3
					
				\end{packed_enum}
			
				\item  \underbar{Aktor 2 (sudionik) može:}
				
				\begin{packed_enum}
					
					\item funkcionalnost 1
					\item funkcionalnost 2
					
				\end{packed_enum}
			\end{packed_enum}
			
			\eject 
			
			
				
			\subsection{Obrasci uporabe}
				
				\textbf{\textit{dio 1. revizije}}
				
				\subsubsection{Opis obrazaca uporabe}
					\textit{Funkcionalne zahtjeve razraditi u obliku obrazaca uporabe. Svaki obrazac je potrebno razraditi prema donjem predlošku. Ukoliko u nekom koraku može doći do odstupanja, potrebno je to odstupanje opisati i po mogućnosti ponuditi rješenje kojim bi se tijek obrasca vratio na osnovni tijek.}\\
					

					\noindent \underbar{\textbf{UC$<$broj obrasca$>$ -$<$ime obrasca$>$}}
					\begin{packed_item}
	
						\item \textbf{Glavni sudionik: }$<$sudionik$>$
						\item  \textbf{Cilj:} $<$cilj$>$
						\item  \textbf{Sudionici:} $<$sudionici$>$
						\item  \textbf{Preduvjet:} $<$preduvjet$>$
						\item  \textbf{Opis osnovnog tijeka:}
						
						\item[] \begin{packed_enum}
	
							\item $<$opis korak jedan$>$
							\item $<$opis korak dva$>$
							\item $<$opis korak tri$>$
							\item $<$opis korak četiri$>$
							\item $<$opis korak pet$>$
						\end{packed_enum}
						
						\item  \textbf{Opis mogućih odstupanja:}
						
						\item[] \begin{packed_item}
	
							\item[2.a] $<$opis mogućeg scenarija odstupanja u koraku 2$>$
							\item[] \begin{packed_enum}
								
								\item $<$opis rješenja mogućeg scenarija korak 1$>$
								\item $<$opis rješenja mogućeg scenarija korak 2$>$
								
							\end{packed_enum}
							\item[2.b] $<$opis mogućeg scenarija odstupanja u koraku 2$>$
							\item[3.a] $<$opis mogućeg scenarija odstupanja  u koraku 3$>$
							
						\end{packed_item}
					\end{packed_item}

				\noindent \underbar{\textbf{UC2 - Pregled profila benda}}
				\begin{packed_item}
	
						\item \textbf{Glavni sudionik: }Javnost
						\item  \textbf{Cilj:} Pregledati profil benda
						\item  \textbf{Sudionici:} baza podataka,bend
						\item  \textbf{Preduvjet:} /
						\item  \textbf{Opis osnovnog tijeka:}
						
						\item[] \begin{packed_enum}
							\item Javnost odabire profil benda koji želi pregledati
							\item Aplikacija prikazuje profil be
							\item Aplikacija korisniku prikaže recenziju eventa
						\end{packed_enum}
				
				\end{packed_item}
				
				
				\noindent \underbar{\textbf{UC3 - Pregled recenzija}}
				\begin{packed_item}
	
						\item \textbf{Glavni sudionik: }Javnost
						\item  \textbf{Cilj:} Omogućiti pregled recenzija javnosti
						\item  \textbf{Sudionici:} baza podataka
						\item  \textbf{Preduvjet:} /
						\item  \textbf{Opis osnovnog tijeka:}
						
						\item[] \begin{packed_enum}
	
							\item Javna osoba pristupi popisu recenzija evenata
							\item Osoba odabere recenziju koju želi vidjeti
							\item Aplikacija korisniku prikaže recenziju eventa
						\end{packed_enum}
				\end{packed_item}

				
				\noindent \underbar{\textbf{UC4 - Registracija}}
				\begin{packed_item}
	
						\item \textbf{Glavni sudionik: }Korisnik
						\item  \textbf{Cilj:} Registrirati se
						\item  \textbf{Sudionici:} Baza podataka
						\item  \textbf{Preduvjet:} /
						\item  \textbf{Opis osnovnog tijeka:}
						
						\item[] \begin{packed_enum}
	
							\item Korisnik odabire opciju za registraciju
							\item Korisnik unosi potrebne korisničke podatke
							\item Korisnik prima obavijest o uspješnoj registraciji
						\end{packed_enum}
						
						\item[] \begin{packed_item}
	
							\item[1] Odabir već zauzetog korisničkog imena i/ili e-maila, unos podataka u        nedozvoljenom format ili pružanje neispravnog e-maila.
							\item[] \begin{packed_enum}
								
								\item  Sustav obavještava korisnika o neuspjelom upisu i vraća ga na stranicu za registraciju.
								\item Korisnik mijenja potrebne podatke ili odustaje od registracije.

							\end{packed_enum}
						\end{packed_item}						
				\end{packed_item}

				\newpage
				\noindent \underbar{\textbf{UC5 - Prijava u sustav}}
				\begin{packed_item}
	
						\item \textbf{Glavni sudionik: } Administrator, korisnik
						\item  \textbf{Cilj:} Korištenje sustava
						\item  \textbf{Sudionici:} baza podataka
						\item  \textbf{Preduvjet:} Autorizacija korisničkog imena i lozinke
						\item  \textbf{Opis osnovnog tijeka:}
						
						\item[] \begin{packed_enum}
	
							\item Korisnik unosi korisničko ime i lozinku
							\item Baza autorizira unesene podatke
							\item[] \begin{packed_enum}
								
								\item Dozvoljava korisniku korištenje sustava ako su podaci ispravni
								\item   Ne dozvoljava korisniku korištenje sustava ako su podaci ne ispravni
								
							\end{packed_enum}
						\end{packed_enum}
				\end{packed_item}


				\noindent \underbar{\textbf{UC6 - Uvid u popis dodanih instrumenata}}
				\begin{packed_item}
	
						\item \textbf{Glavni sudionik: }Administrator
						\item  \textbf{Cilj:} Pregled dodanih instrumenata od strane glazbenika.Uklanjanje nepotrebnih zapisa u bazi podataka (npr. "Gitara", "gitara")
						\item  \textbf{Sudionici:} baza podataka
						\item  \textbf{Preduvjet:} Glazbenici su dodali svoje instrumente i ti instrumenti su zapisani u bazi podataka
						\item  \textbf{Opis osnovnog tijeka:}
						
						\item[] \begin{packed_enum}
	
							\item Administrator odabire pregled dodanih instrumenata
							\item Ispis dodanih instrumenata
						\end{packed_enum}
				\end{packed_item}


				\noindent \underbar{\textbf{UC9 - Slanje poruka}}
				\begin{packed_item}

					\item \textbf{Glavni sudionik: }Administrator, glazbenik, korisnik
					\item  \textbf{Cilj:} Komunnikacija između korisnika
					\item  \textbf{Sudionici:} Baza podataka
					\item  \textbf{Preduvjet:} Korisnik je prijavljen
					\item  \textbf{Opis osnovnog tijeka:}
					
					\item[] \begin{packed_enum}

							\item Korisnik odabire "poruke"
							\item Aplikacija prikazuje pregled osoba s kojima se već vodio razgovor
							\item Korisnik odabire postojeći razgovor i prikazuju mu se prošle poruke
							\item Korisnik odabire "nova poruka" te pronalazi korisnika kojemu želi poslati poruku
							\item Korisnik odabire polje za pisanje poruke
							\item Korisnik napiše željenu poruku
							\item Korisnik odabere "Pošalji" za slanje poruke
					\end{packed_enum}
				
					\item  \textbf{Opis mogućih odstupanja:}
				
					\item[] \begin{packed_item}

						\item[2.a] Korisnik ima novih poruka
						\item[] \begin{packed_enum}
							\item Ukoliko postoji nova poruka, ime pošiljatelja je podebljano
						\end{packed_enum}
					\end{packed_item}
				\end{packed_item}
			
			\noindent \underbar{\textbf{UC10 - Pisanje recenzija o bendovima}}
			\begin{packed_item}
				
				\item \textbf{Glavni sudionik: }Korisnik
				\item  \textbf{Cilj:} Omogućiti korisnicima aplikacije pisanje recenzija o bendovima
				\item  \textbf{Sudionici:} Baza podataka, bend
				\item  \textbf{Preduvjet:} Korisnik mora biti prijavljen
				\item  \textbf{Opis osnovnog tijeka:} 
				
				\item[] \begin{packed_enum}
					
					\item Korisnik pristupa profilu benda
					\item U okvir za poruku napiše svoje mišljenje o bendu te označi broj zvjezdica kao ocjenu
					\item Korisnik odabere "Završi recenziju"
				\end{packed_enum}
			\end{packed_item}
				
			\noindent \underbar{\textbf{UC11 - Pisanje recenzija za događaje}}
			\begin{packed_item}
				
				\item \textbf{Glavni sudionik: }Korisnik
				\item  \textbf{Cilj:} Napisati recenziju za događaj
				\item  \textbf{Sudionici:} Baza podataka
				\item  \textbf{Preduvjet:} Korisnik je prijavljen
				\item  \textbf{Opis osnovnog tijeka:} 
				
				\item[] \begin{packed_enum}
					
					\item Korisnik odabire opciju za recenziranje događaja
					\item Otvara se prozor za unos recenzije
					\item Korisnik unosi recenziju i potvrđuje se
					\item Recenzija se bilježi na stranici događaja
				\end{packed_enum}
			\end{packed_item}
		 
		    \noindent \underbar{\textbf{UC12 - Pregled profila glazbenika}}
		    \begin{packed_item}
		    	
		    	\item \textbf{Glavni sudionik: }Korisnik, glazbenik
		    	\item  \textbf{Cilj:} Pregled profilne stranice glazbenika 
		    	\item  \textbf{Sudionici:} Baza podataka
		    	\item  \textbf{Preduvjet:} Autorizacija korisničkog imena i lozinke
		    	\item  \textbf{Opis osnovnog tijeka:} 
		    	
		    	\item[] \begin{packed_enum}
		    		
		    		\item Korisnik odabire prikaz profila glazbenika
		    		\item Prikazuju mu se javni podaci glazbenika (kalendar, instrumenti, popis nastupa na kojima svira)
		    	\end{packed_enum}
		    \end{packed_item}
	    
	    	 \noindent \underbar{\textbf{UC13 - Komentiranje objava glazbenika}}
	    	\begin{packed_item}
	    		
	    		\item \textbf{Glavni sudionik: } Korisnik
	    		\item  \textbf{Cilj:} Poboljšati interakciju između korisnika i glazbenika 
	    		\item  \textbf{Sudionici:} Baza podataka, glazbenik
	    		\item  \textbf{Preduvjet:} Korisnik je prijavljen i glazbenik je stvorio objavu
	    		\item  \textbf{Opis osnovnog tijeka:} 
	    		
	    		\item[] \begin{packed_enum}
	    			
	    			\item Glazbenik napisao objavu
	    			\item Korisnici koji prate tog glazbenika vide napisanu objavu te biraju opciju "Komentiraj"
	    			\item Korisnik piše komentar
	    			\item Korisnik odabere "Spremi"
	    			\item Objava je vidljiva glazbeniku
	    		\end{packed_enum}
	    	\end{packed_item}

			\noindent \underbar{\textbf{UC17 - Stvaranje profila glazbenika}}
			\begin{packed_item}
				
				\item \textbf{Glavni sudionik: } Korisnik
				\item  \textbf{Cilj:} Omogućiti korisniku da se registrira kao glazbenik 
				\item  \textbf{Sudionici:} Baza podataka, glazbenik
				\item  \textbf{Preduvjet:} Korisnik je prijavljen
				\item  \textbf{Opis osnovnog tijeka:} 
				
				\item[] \begin{packed_enum}
					
					\item Korisnik zatraži stvaranje profila glazbenika na svome profilu
					\item Korisnik ispuni tražene podatke
					\item Korisnik odabere “Stvori profil glazbenika”
			
				\end{packed_enum}
			\end{packed_item}	
		
		   \noindent \underbar{\textbf{UC18 - Odjava}}
		   \begin{packed_item}
		   	
		   	\item \textbf{Glavni sudionik: } Korisnik
		   	\item  \textbf{Cilj:} Odjaviti se
		   	\item  \textbf{Sudionici:} Baza podataka
		   	\item  \textbf{Preduvjet:} Korisnik je prijavljen
		   	\item  \textbf{Opis osnovnog tijeka:} 
		   	
		   	\item[] \begin{packed_enum}
		   		
		   		\item Korisnik odabire iz padajućeg izbornika opciju za odjavu
		   		\item Korisnik se uspješno odjavljuje iz aplikacije
		   		
		   	\end{packed_enum}
		   \end{packed_item}	 
		
		
				
				\subsubsection{Dijagrami obrazaca uporabe}
				
				
				
					
					\textit{Prikazati odnos aktora i obrazaca uporabe odgovarajućim UML dijagramom. Nije nužno nacrtati sve na jednom dijagramu. Modelirati po razinama apstrakcije i skupovima srodnih funkcionalnosti.}
				\eject				
				
				
			\subsection{Sekvencijski dijagrami}
				
				\textbf{\textit{dio 1. revizije}}\\
				
				\textit{Nacrtati sekvencijske dijagrame koji modeliraju najvažnije dijelove sustava (max. 4 dijagrama). Ukoliko postoji nedoumica oko odabira, razjasniti s asistentom. Uz svaki dijagram napisati detaljni opis dijagrama.}
				\eject
	
		\section{Ostali zahtjevi}
		
			\textbf{\textit{dio 1. revizije}}\\
		 
			 \textit{Nefunkcionalni zahtjevi i zahtjevi domene primjene dopunjuju funkcionalne zahtjeve. Oni opisuju \textbf{kako se sustav treba ponašati} i koja \textbf{ograničenja} treba poštivati (performanse, korisničko iskustvo, pouzdanost, standardi kvalitete, sigurnost...). Primjeri takvih zahtjeva u Vašem projektu mogu biti: podržani jezici korisničkog sučelja, vrijeme odziva, najveći mogući podržani broj korisnika, podržane web/mobilne platforme, razina zaštite (protokoli komunikacije, kriptiranje...)... Svaki takav zahtjev potrebno je navesti u jednoj ili dvije rečenice.}
			 
			 
			 
	