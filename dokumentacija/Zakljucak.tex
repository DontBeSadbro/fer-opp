\label{key}\chapter{Zaključak i budući rad}
		
		
		
	
		 
		 Zadatak našeg projekta je bio razvoj web aplikacije koja bi olakšala praćenje i stvaranje muzičkih događaja, uglavnom koncerata. Namijenjena je glazbenicima, organizatorima te javnosti. SVaka osoba se može registrirati i prijaviti, te u svojem profilu naznačiti da je glazbenik i/ili organizator. Glazbenici mogu stvoriti profil benda te pozvati druge glazbenike u bend, dok organizatori mogu stvoriti događaje te pozvati bend da nastupa na događaju. Javnost ima pristup popisu događaja te njihovim detaljima, kao i prikaz lokacije događaja na Google Maps-u. Aplikacija korisnicima nudi i ostale funkcionalnosti, kao što su poruke, komentari, objave itd.
		 
		 Prva faza projekta je uključivala međusobno upoznavanje ljudi u grupi i dogovor oko zadatka projekta koji se sastojao od generalnog opisa aplikacije. Nakon prvog sastanka sa asistentom te demonstratorom, te nakon što nam je odobren naš prijedlog zadatka, naša grupa je, uz učestale sastanke sa asistentom, počela ulaziti u detalje razrade samog zadatka. Nakon što je zadatak bio razrađen, i nakon naučenog gradiva sa predavanja kolegija, počeo je i rad na dokumentaciji. Taj rad se sastojao od opisa zadataka, navođenja specifikacija programske potpore(funkcionalni te ostali zahtjevi) te opisa arhitekture i raznih dijagrama. Na taj način smo stvorili "kostur" aplikacije, te nam je početak samog programskog rješavanja zadatka zbog toga bio puno lakši. Za kraj prve faze, naša aplikacija je imala funkcionalnu registraciju te prijavu korisnika.
		 
		 Druga faza projekta je u početku imala glavni fokus na programskoj implementaciji svih zahtjeva. Obzirom da 5 od 7 članova grupe nije imala nekakvog prethodnog iskustva sa radom na aplikacijama, ova faza je bila zahtjevnija, kompliciranija te izazovnija. Članovi grupe su se morali učiti nove vještine kako bi uspijeli implementirati funkcionalnu aplikaciju. Iskusniji članovi su pomogli manje iskusnima te ih naučili nešto novo, ili u krajnjem slučaju ih uputili korisnim resursima na internetu. Što se dokumentacije tiče, za drugu fazu smo dokumentaciju morali nadopuniti sa više dijagrama koji detaljnije opisuju naš zadatak, te informacijama vezane za alate koje smo koristili, ispitivanje sustava te uputa za puštanje aplikacije u pogon.
		 
		 Kroz obje faze projekta su članovi tima komunicirali preko platforme Slack, preko koje su se i dogovarali sastanci uživo kako bi se lakše dogovorili oko određenih stvari te radi lakše i bolje komunikacije općenito. Isto tako, redovitim sastancima sa asistentom se uvelike poboljšao uvid u zadatak svakog člana.
		 
		 Za završetak, naša grupa nije uspijela implementirati sve funkcionalne zahtjeve, ali svi zahtjevi visokog te skorosvi zahtjevi srednjeg prioriteta su implementirani. Grupa je naišla na brojne prepreke različitih oblika tijekom razvoja aplikacije, od malog predznanja i vremena obzirom na ostale obaveze na fakultetu, kompleksnih prepreka vezane za implementacije određenog dijela funkcionalnosti unutar aplikacije do problema koji su bili van naše moći, kao naprimjer greška u najnovijoj inačici nužnog paketa za frontend. 
		 
		 Bez obzira na sve probleme koji su se pojavili u vremenu projektnog zadatka, članovi grupe su generalno zadovoljni rješenjem zadatka. Uz to što su svi naučili nešto novood strane tehnologija, programiranja te pisanja dokumentacije, članovi su stekli i nova poznanstva te su naučili kako komunicirati i raditi kao tim.
		 
		 
		
		\eject 
		
	